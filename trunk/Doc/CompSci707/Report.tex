\documentclass{article}
\begin{document}
\title{\bfseries A Scene Graph Implementation}
\date{\today}
\author{Sashan Govender \\
Student ID: 3477445 \\
CompSci707 - Project} 
\maketitle
\begin{abstract}
This project implements a basic scene graph. A scene graph is an data structure
for organizing the graphical data, like meshes, lights and materials. It allows
one to separate the details of the rendering from the data that is being rendered.
The scene graph implemented is a DAG (directed acyclic graph). This means that each node can
have only 1 parent. Some scene graphs, like the Maya scene graph, allow for more than one parent
per node.  
\end{abstract}
\section{Goals}
The initial goal was to implement a basic scene graph library capable of
rendering textures, lighting and displaying meshes. Since there are already many
implementations, I felt that to make it different it would be good to try and
emulate the concepts ebodied in the C++ STL. The C++ STL is striking because it
diverges from traditional object oriented design, where concepts are related by
an inheritence hierarchy. It is easy to abuse this and force an inheritance
structure where there should not be one. It is also easy force a hideous number
of member functions into a class. The STL approach is more like a 'whole-part'
relation. The whole object is composed of distinct parts. There need not be an
iheritance structure between the parts and the object as a whole. Another key is
that data structures and algorithms are seperated. The separation is achieved
using iterators that decouple the algorithm from the data structure. In order to
achieve this, where the scene graph is essentially a data structure, with
algorithms separated, I decided that it should be implemented using the boost
graph library. Unfortunately this didn't work out as planned and the problem (a
user interface problem) is described later.
\section{Features}
The scene graph has classes that represent the following concepts that are
common when programming computer graphics applications.
\begin{itemize}
	\item Translation
    \item Rotation
    \item Scaling
    \item Texture    
    \item Material
    \item Mesh
    \item Point Light
    \item Direction Light     
\end{itemize}
\section{Architecture}
\subsection{The Node class}
All elements that can be members of the scene graph inherit from a base class
called Node. The Node class implements a container for it's children using
std::list, as shown:
\newline
\verb|std::list<NodePtr> m_Children;|
\newline
Building the scene graph is a matter of defining a root node and then attaching
children to it. For example:
\begin{verbatim}
Node::NodePtr g_pRoot(new Node);
Translate::TranslatePtr pTrans(new Translate);
Mesh::MeshPtr pMesh(new Mesh);
pTrans->AddChild(pMesh);
g_pRoot->AddChild(pTrans);	
\end{verbatim}
\subsection{Traversing the scene graph}
There are two classes used to traverse the scene graph. All traversal is done
using a depth first search. One is PreRenderVistior and the other is
RenderVisitor. PreRenderVisitor is meant to be run only once on the scene graph
during the entire application. It does the initializing work, like loading and
generating textures. RenderVisitor is run on the graph whenever the display
needs updating. It's important to note that \textbf{none} of the classes derived from
Node, like Translate, Rotate, Material etc... contain any information about the
API doing the rendering. In other words there are no calls OpenGL calls in any
of those classes. All the knowledge about OpenGL is separated into RenderVisitor
and PreRenderVisitor. This makes it easy to add another class, say
DirectGraphicsVisitor, that traverses the graph and renders using
DirectGraphics, without having to modify the classes derived from Node.
\subsection{Meshes, Materials and TextureMaps}
The surface color of a Mesh is defined by the Material or TextureMap nodes that
parent it. Initially it might make more sense to make Material and TextureMap
objects member variables of a mesh class. But what happens when the same
Material or TextureMap is used by several meshes. Then the calls to
glBindTexture will be repeated per mesh. This is inefficient since the graphics
card has to constantly shift textures in and out of memory. It would be better
if all the meshes that used the same texture were drawn while that texture was
in the cards memory. Making the TextureMap class a parent Node of the Mesh and
not a member variable allows the system to accomplish this.
\subsection{Smart Pointers}
All classes that can be elements of the scene graph have a smart pointer
implementation using \verb|boost::shared_ptr|. The problems with using raw pointers in
conjunction with container classes is well known (memory leakage) and using a
smart pointer implementation avoids the memory leak problem. However it does
introduce other problems. For example consider dynamic casting. This is easy to
accomplish with a raw pointer:
\newline
\verb|Derived* pD = dynamic_cast<Derived*>(pBase)|
\newline
However when using a \verb|shared_ptr| that wraps the raw pointer, using
\verb|dynamic_cast| isn't going to help. Fortunately the Boost smart pointer
library provides a casting function \verb|dynamic_pointer_cast| that handles objects 
of type \verb|shared_ptr<T>|. It is used exactly like an ordinary dynamic cast:
\newline
\verb|shared_ptr<Derived> pD = dynamic_pointer_cast<Derived>(pBase)|
\newline
\paragraph{}
Another issue faced was how to deal with the \verb|this| pointer when the object
that calls the member function is a shared pointer. Often we'd like to convert the this
pointer into it's corresponding shared pointer. The following case makes the
problem faced apparent. Consider the case of generating textures. When
the PreRenderVisitor encounters a TextureMap it generates a texture for it. The
initial version of the function was like this:
\begin{verbatim}
void TextureMap::ApplyVisitor(Edge::NodeVisitor* pVisitor)
{
    PreRenderVisitor* pOGLTexGen = dynamic_cast<PreRenderVisitor*>(pVisitor);
    if (pOGLTexGen)
    {
        pOGLTexGen->GenerateTexture(this);
    }
    Node::ApplyVisitor(pVisitor);
}
\end{verbatim}
This passes a raw pointer to the member function \verb|GenerateTexture|.
\begin{verbatim}
void PreRenderVisitor::GenerateTexture(TextureMap* pTexMap)
{
    GLuint ID;	
    glGenTextures(1, &ID);
    m_IDToTexture[ID] = pTexMap;
}
\end{verbatim}
\verb|m_IDToTexture| is just map that maps from the ID to the corresponding
texture. As you can see by doing it like this a raw pointer unreferenced pointer
is stored. This violates our rule stating that it's bad to place raw pointers in
container classes. It's especially bad in this case because pTexMap is uncounted
for by the smart pointer system. The solution is to make the interface
to \verb|GenerateTexture| require a smart pointer:
\begin{verbatim}
void PreRenderVisitor::GenerateTexture(TextureMapPtr pTexMap)
\end{verbatim}
Now the problem is how to convert the this pointer into a it's corresponding
smart pointer. You could force ApplyVisitor to pass the this smart pointer
explicitly 
\begin{verbatim}
void TextureMap::ApplyVisitor(Edge::NodeVisitor* pVisitor, TextureMapPtr pThis)
{
    PreRenderVisitor* pOGLTexGen = dynamic_cast<PreRenderVisitor*>(pVisitor);
    if (pOGLTexGen)
    {
        pOGLTexGen->GenerateTexture(pThis);
    }
    Node::ApplyVisitor(pVisitor);
}
\end{verbatim}
but this solution is ugly. Fortunately the smart pointer implementation has a class 
\verb|enable_shared_from_this| that allows a class to generate a smart pointer
to itself. The \verb|Node| class inherits from \verb|enable_shared_from_this|
for this reason. 
\begin{verbatim}
class Node : public boost::enable_shared_from_this<Node>
\end{verbatim}
The final implementation of \verb|ApplyVisitor| is
\begin{verbatim}
void TextureMap::ApplyVisitor(Edge::NodeVisitor* pVisitor)
{
    PreRenderVisitor* pOGLTexGen = dynamic_cast<PreRenderVisitor*>(pVisitor);
    if (pOGLTexGen)
    {
        TextureMapPtr pThis(dynamic_pointer_cast<TextureMap>(shared_from_this()));
        pOGLTexGen->GenerateTexture(pThis);
    }
    Node::ApplyVisitor(pVisitor);
}  
\end{verbatim}
\section{Problems}
\subsection{The Boost Graph Library (BGL)}
To explain the user interface type problem I had with the BGL it's first
important to discuss the basics of the BGL. The BGL allows one to create a type
to represent a graph. It is highly flexible and complicated. It allows one to
choose between a directed or undirected graph, the container classes used to
hold vertices, the container class used to hold edges and change the properties of the vertices
and edges. It provides various algorithms for traversing the graph and finding the 
shortest path between vertices.  
To construct a graph type:
\begin{verbatim}
typedef adjacency_list < vecS, vecS, directedS > graph_t;
\end{verbatim}
This means make a graph using std:vector as the edge container and std::list as
the vertex container. \verb|vecS| is an identifier defined in the BGL headers to
mean std::vector. You can traverse the graph like this:
\begin{verbatim}
graph_t g;
depth_first_search(g, visitor(OGLVisitor));
\end{verbatim}
If the graph type is changed to use std::list as the vertex container class
\begin{verbatim}
typedef adjacency_list < vecS, listS, directedS > graph_t;
\end{verbatim}
then this doesn't compile
\begin{verbatim}
graph_t g;
depth_first_search(g, visitor(OGLVisitor));
\end{verbatim}
The solution is to manually assign indices to the \verb|vertex_index| property
of the graph like this:
\begin{verbatim}
int main()
{	    
     typedef adjacency_list < vecS, listS, directedS,
        property< vertex_index_t, int> > graph_t;

     graph_t g;
     OpenGLVisitor OGLVisitor;
     graph_t::vertex_iterator i, e;
     int id(0);
     for(tie(i, e) = vertices(g); i != e; ++i, ++id)
        put(vertex_index, g, *i, id);
     depth_first_search(g, visitor(OGLVisitor));	    
    return 0;        
}    
\end{verbatim}
I felt that this was an artificial solution and not very nice behaviour so I
stopped using the BGL.  
For more information see the thread at 
http://article.gmane.org/gmane.comp.lib.boost.devel/26501/
\section{Sample Applications}
There are two sample applications. They can be found in the WinApp directory. In
the texture demo pressing 'm' changes to \verb|GL_MODULATE| and 'r' changes to
\verb|GL_REPLACE|.
\section{Documentation}
Documentaion was generated using doxygen. See the directory \verb|Doc\html|.
There wasn't enough time to document all the classes thoroughly but you can get
an overview of the class hierarchy and relationships.
\section{Conclusion}
The scene graph can be extended to support bump maps and specular maps. The
video cards on the university computers don't support these basic features.
In hindsight I would not have tried to use the BGL given the time constraint.
Future work could move the Translate, Rotate and Scale nodes into a
TransformGroup node and the TextureMap and Material nodes can become part of a
ShaderGroup node.  
\end{document}

