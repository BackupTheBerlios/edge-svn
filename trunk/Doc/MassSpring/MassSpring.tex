
\documentclass{article}

\begin{document}
Equations of motion for a single particle mass connected by two springs. The springs connect the mass to 
two fixed points. We know the following.
\begin{itemize}
\item $k_{0}$ = the spring constant of spring 0.
\item $k_{1}$ = the spring constant of spring 1.
\item $m$ = the mass of the particle.
\end{itemize}
We have to find an equation for:
\begin{itemize}
\item $x(t)$ = the displacement of the particle from it's resting position.
\end{itemize}
Let $F_{0}$ be the force on the particle due to spring 0.
\begin{equation}
	F_{0} = -k_{0}x
\end{equation}
Let $F_{1}$ be the force on the particle due to spring 1.
\begin{equation}
	F_{1} = -k_{1}x
\end{equation}
Then the total force on the particle is
\begin{equation}
	F = F_{0} + F_{1}
\end{equation}
and using Newton's law $F = ma$
\[F = (-k_{0} - k_{1})x = m\ddot{x}	\]
\[\ddot{x} + \frac{(k_{0} + k_{1})}{m}x = 0\]
Letting $K = \frac{(k_{0} + k_{1})}{m}$
\begin{equation}
	\ddot{x} + Kx = 0
\end{equation}
This is a standard 2nd order ODE that we can solve by letting $x = e^{rt}$.
Then $\ddot{x} = r^{2}e^{rt} = r^{2}x$ Therefore:
\[
	r^{2}x + Kx = 0
\]
and solving for r gives:
\[r = \pm\sqrt{-K} = \pm i \sqrt{K}\]
Therefore a general solution is:
\begin{equation}
	x(t) = Ae^{i\sqrt{K}t} + Be^{-i\sqrt{K}t}	
\end{equation}
and by using $e^{i\theta} = cos(\theta) + isin(\theta)$ can also be written as
\begin{equation}
	x(t) = Ccos(\sqrt{K}t) + Dsin(\sqrt{K}t)
\end{equation}
\end{document}

