\chapter{Constrained Systems: Penalty Method}
\label{Cha:PenaltyMethod}
A penalty method implements a constraint on a particle by imposing a force on a  
particle when it breaks a condition. This force may not exist in real life. For
example consider simulating a ball hitting the floor. A penalty method will let the ball
pass through the ground and the create a reactionary force with a magnitude
proportional to the penetration depth and direction in the direction of the
surface normal of the ground. Physically this is incorrect since the ball never
passes through the ground. However the visual result, which is what we are
concerned with here since this is computer graphics, is realistic enough. One
way to implement a penalty method is by using a mass spring system.
\section{Generalized Mass Spring System}
\label{sec:massspring}
For a generalized mass spring system the goal is to work out the position of
each particle at a given time. Using Newtonian mechanics this means we need to
know the acceleration of each particle. The acceleration can be computed once we
know the force on a particle. The force on a particle is the sum of the forces
generated by the joining springs. Since we know how to compute the force
generated by a spring (the spring constant multiplied by its displacement from
resting length) we can compute the force on a particle. Hence the equation for
the magnitude of the force on a particle $i$ is
\begin{equation}
   \label{magforce}
	F_i = \sum_{j=0}^{n}k_j\left(\left|{x}_i -
      {x}_j\right|-l_j\right)        
\end{equation}
where $n$ is the number of connecting springs, $k_j$ is the spring constant and
$l_j$ is the rest length of the spring. In order to make this usable the
direction that the force is acting in needs to be considered. Representing the
position of the particle by the vector 
\[
    \mathbf{x} = \begin{bmatrix}
                    x\\
                    y\\
                    z
                 \end{bmatrix}
\] 
Then the direction vector between two particles is
\[
    \mathbf{r}_{ij} = \mathbf{x}_j - \mathbf{x}_i
\]
and the normalized direction vector is 
\[
    \frac{\mathbf{r}_{ij}}{\left|\mathbf{r}_{ij}\right|}
\]
Now that we have a direction in which to apply the magnitude of the force,
that won't alter the final magnitude, since it is normalized, an  
equation for the force on particle $i$ in vector notation is 
\begin{equation}
\label{Eqn:UndampedSpring}
    \mathbf{F}_i = \sum_{j=0}^{n}k_j\left(\left|\mathbf{r}_{ij}\right|-l_j\right)
                    \frac{\mathbf{r}_{ij}}{\left|\mathbf{r}_{ij}\right|}
\end{equation} 
When simulating systems on a computer inaccuracies accumulate as a result of the
the numerical methods used (see Appendix \ref{Cha:DESolvers}). This results can
result in system instability where energy is added to the system. The energy of
a perfect system should remain constant. In reality most systems will lose
energy due to friction. For the mass spring system some damping terms are
introduced to give the final equation. The damping takes the form of a force
opposing the motion of the particle and proportional to the velocity as well as
a force proportional to the rate of change of the spring length. A common way to
model damping is to add an opposing force that has a magnitude proportional to
the rate of change of some property. In the case of the movement of the particle
through the air the property is distance, hence equation \ref{Eqn:UndampedSpring}
becomes
\begin{equation}
    \label{Eqn:DampedSpring}
      \mathbf{F}_i = \sum_{j=0}^{n}k_j\left(\left|\mathbf{r}_{ij}\right|-l_j 
                    + k_d\frac{d}{dt}\left|\mathbf{r}_{ij}\right|\right)
                    \frac{\mathbf{r}_{ij}}{\left|\mathbf{r}_{ij}\right|} 
\end{equation}
where $k_d$ is the constant of damping that we are free to set to get the
desired behaviour.

\section{Implementation}
The implementation allowed for an arbitrary configuration of masses and springs.
Initially the Boost Graph Library (BGL) \cite{BGL} was used. Nodes in the graph
stored references to the particles and the edges of the graph stored references
to the springs. After a while the BGL was removed because it was felt to be
overweight for the needs of a mass spring system. For example there was no need
for a shortest path algorithm. Indeed, an interesting question is what physical
meaning can be attributed to the shortest path between 2 particles in the graph,
where the potential energy of the spring gives the edge length.

An algorithm and data structure for computing the forces on all the particles is
given below. It comes from \ref{Eqn:DampedSpring}, which gives the force on a
particle. The data structure allows for a mass spring system of any topology to
be constructed. 

The data structure is a container with triple element particle spring particle 

\newcounter{count0}
\begin{list}{\arabic{count0}}{\usecounter{count0}}
    \item{For every particle}
    \newcounter{count1}
    \begin{list}{\arabic{count0}.\arabic{count1}}{\usecounter{count1}}
        \item{For every spring connected to this particle}
        \newcounter{count2}
        \begin{list}{\arabic{count0}.\arabic{count1}.\arabic{count2}}{\usecounter{count2}}
            \item{Calculate the length the spring is stretched ($\vert \mathbf{r}_{ij} \vert - l_j$)}                                    
            \item{For damping calculate the relative speeds of the particles and multiply by the damping constant($k_d\frac{d}{dt}\left|\mathbf{r}_{ij}\right|$)}                        
            \item{Calculate the normalized direction
            ($\frac{\mathbf{r}_{ij}}{\vert \mathbf{r}_{ij} \vert}$)}
            \item{The force on the particle due to the given spring is the sum 
            of the above 2 terms multiplied by the spring constant $k_j$. 
            The direction of the force is given by the normalized direction vector.}            
        \end{list}
    \end{list}    
\end{list}

\section{Summary}
The mass spring system is a versatile model for producing realistic looking
simulations. With a little imagination it can be applied to simulate a variety
of phenomenon like cloth and rigid links between between particles. It
also interacts well with other bodies that might be in the simlutation.
Stiffness \ref{Cha:DESolvers} is a problem to watch out for.
