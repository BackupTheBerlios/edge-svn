\chapter{Conclusion}
The physics system implemented only dealt with the dynamics of constrained and
unconstrained particles. It could have benefitted from a more generic collision
detection system as well as support for rotational dynamics and rigid body
dynamics. The solution to the equation \ref{Eqn:LambdaStable} is found using a
bi-conjugate gradient solver. The sparse matrix implementation used comes from
Boost uBLAS
\footnote{\url{http://www.boost.org/libs/numeric/ublas/doc/index.htm}}. Formula
for various constraints were derived and implemented. A scene graph was
implemented and a method for traversing the graph without using runtime
polymorphism (i.e. virtual functions) to perform operations such as rendering
was described and implemented.

\section{Future Work}
The implementation modelled a particle system only. A system that handles
rotational dynamics with constraints would be nice. A solution for controlling
the hexapod model using a genetic algorithm or neural network would be nice. A
better model for friction perhaps using the complicated paper on friction
proposed by Baraff in \cite{BaraffFriction}. A numerical differential equation
solver with adaptive step sizes could be used and the benefits in accuracy and
speed evaluated by comparing with Mathematica's output. The simulation would
have benefitted from more robust collision detection using a broad phase
analysis such as bounding volumes. 
