\chapter{Introduction}
\label{sec:Introduction}
\section{Context Information}
\label{Sec:ContextInformation}
This thesis is part of the field in computer graphics that deals with
physically based animation. Physically based animation deals with the movement
of particles and rigid bodies according to the laws of physics and how these
particles change with time. The book Physics for Game developers \cite{Bourg}
gives an overview of physics applied to games.  More advanced mechanics are
covered in the SIGGRAPH course notes by Witkin and Baraff \cite{PBMNotes}, in
Otte's thesis \cite{Otte} and in Goldstein's book on classical mechanics
\cite{Goldstein}.  A method for setting up a constrained dynamics system based
on the SIGGRAPH course notes \cite{PBMNotes} is described and implemented. An
alternative approach using the Lagrangian dynamics to specify constraints was
studied. Other alternative methods for rigid body simulation include the
Featherstone technique \cite{Featherstone} where a notation, \textit{spatial
notation}, is introduced that greatly simplifies the analysis of rigid body
dynamics. In \cite{Hahn} a technique using collision detection and response is
used to simulate the motion of rigid bodies. A similar technique,
\textit{impulse-based simulation}, is used in \cite{Mirtich}. Barr et al.
\cite{BarzelBarr} describes a method for constrained dynamics similar to
\cite{PBMNotes}, the method used to implement the dynamics system in this
thesis.

Control of a rigid body, such as a humanoid or hexapod, is a field of active
research. McKenna et al. \cite{McKennaZeltzer} describes a system to control an
insect. The system is divided into into a gait controller, motor controller and
dynamics system.  At the University of Auckland Russel Smith devised a neural
network system for control of an artificial robot \cite{RusselSmith}. Karl Sims
describes a technique to evolve controllers for virtual creatures in a
simulated three dimensional physical world using genetic algorithms
\cite{Sims}. The physical simulation, like \cite{McKennaZeltzer}, uses
Featherstone's algorithm. Sanders et al.  \cite{SandersLobbRiddle} describes an
evolutionary algorithm for the locomotion of a two dimensional creature made
out of a mass spring system.  Some impressive work in composing controllers is
described by Faloutsos in \cite{FaloutsosvandePanneTerzopolous}. Composition of
controllers involves deciding when to transition from a controller that
describes, for example, a running gait to a controller that describes a walking
gait. The learning algorithm used is known as \textit{Support Vector Machines}
(SVM) and is a method for fitting functions to an existing set of training
data. For more information on SVMs see \cite{BennetCampbell, Burges}.

\section{Goals}
\label{Sec:Goals}
The goals of this thesis were the following:
\begin{itemize}
\item To implement a physically based animation engine that could handle
constrained particle dynamics.
\item To test the implementation on simple systems like a pendulum and hence
verify its correctness.
\item To construct a model of a hexapod using the constrained dynamics system.
\item To investigate control of the hexapod under the system.
\item To investigate constrained dynamics using Lagrangian methods as an
alternative to the Newtonian based method described by Witkin and Baraff
\cite{PBMNotes}  
\end{itemize}

\section{Tools Used}
\label{Sec:Tools}
The system was written using Visual C++ 7.1. The Boost \footnote{see \url{http://www.boost.org}} packages
used were the unit testing package, the Boost implementation of UBLAS, the
Boost Graph Library and the shared pointer library. Bitkeeper was used for
revision control. OpenGL was used to render the scene. The STL, specifically
the container classes vector, list and map, were heavily used. Source code for
this thesis can be found at \url{http://edge.bkbits.net}. 
The computer systems used were a Pentium 4 with 2.8
Ghz running Windows XP and a AMD 1.5 GHz machine running Windows XP and Linux.
The majority of this thesis was produced using Latex on Linux using Dia \footnote{\url{http://www.gnome.org/projects/dia/}}
for the diagrams.

\section{Chapter Overview}
\label{Sec:ChapterOverview}
\subsection{Chapter 2}
This chapter discusses physically based animation. Animation is discussed in
terms of Newtonian dynamics. Constrained dynamics are covered in detail and
various constraints are evaluated. A section on Lagrangian dynamics is included.

\subsection{Chapter 3}
This chapter covers the scene graph implementation and how the constrained dyanmics
system is implemented. It also describes a technique that can eliminate messy
dynamic casting from base to child nodes when traversing a scene graph. Details
of how the system was verified are also given.

\subsection{Chapter 4} 
Here the implementation of the hexapod and details of the
attempts at making it move are mentioned.

\subsection{Chapter 5}
This is the conclusion in which a summary of the results and suggestions
for future work are given.

\subsection{Appendix A}
Here we present an introduction to numerical differential equation solvers. It
covers the Euler method, Midpoint method and the Runge-Kutta 4th order method.

\subsection{Appendix B}
A simple proof that is used in the derivation of formulae in
the constrained dynamics section in Chapter 2 is given.


