\chapter{Introduction to Lagrangian Dynamics}
\textit{Where the Lagrangian comes from, derives the function (a straight line)
for the path between two points on a plane using variational calculus, discusses
a simple spring mass system and derives the equations of motion using Lagrangian
and Newtonian methods.}
\section{Variational Calculus}
The calculus of variations is concerned with finding a value for the function,
y(x), that will minimize or maximize (extremize) another function, F. In a
sense, variational calculus is a generalization of calculus itself since in
ordinary calculus one of the problems is to find a value for x that will
extremize y(x). The explanation that follows is based on \cite{DurrantFraser}.
\subsection{Path Between Two Points on a Plane}
\label{sec:PathBetweenTwoPointsOnAPlane}
Imagine that we have two points on a plane and we want to find the function that
describes the path that minimizes the distance between them. Naturally the
shortest path between them is given by a straight line equation, however it will
be shown how that equation can be derived using techniques from the calculus of
variations. Let $\Delta s$ = a small displacement along the path. Then, from
Pythagoras, $\Delta s^2 = \Delta x^2 + \Delta y^2$.
\begin{eqnarray*}
	\frac{\Delta s^2}{\Delta x^2} = 1 + \frac{\Delta y^2}{\Delta x^2}\\
	\frac{\Delta s}{\Delta x} = \sqrt{1 + \frac{\Delta y^2}{\Delta x^2}}
\end{eqnarray*}
Taking the limit as $\Delta x$ tends to $0$ we get
\begin{equation}
	\frac{ds}{dx} = \sqrt{1 + \dot{y}^2}
\end{equation}
If we sum all the infinitesimal changes in the displacement we'll get the
distance, D, between the points on the plane.
\begin{equation}
	\label{f_to_minimize}
	D = \int_{0}^{D} ds = \int_{x_0}^{x_1} \sqrt{1 + \dot{y}^2} dx
\end{equation}
Now our task is to find the function y, that will minimize D. Fortunately it has
been proved \cite[pg 37 - 39]{Goldstein} that a necessary condition for a function, $y(x)$, to extremize the
above integral is:
\begin{equation}
	\label{Eqn:EulerLagrange}
	\frac{\partial F}{\partial y} - \frac{d}{dx}\left(\frac{\partial F}{\partial \dot{y}}\right) = 0
\end{equation}
where in general $F = F(x,y(x),\dot{y}(x))$
\ref{Eqn:EulerLagrange} is called the Euler-Lagrange equation.
In the case above the function to extremize is:
\begin{equation}
F = F(\dot{y}) = \sqrt{1 + \dot{y}^2}
\end{equation}
and by using \ref{euler_lagrange} an expression for $y(x)$ can be obtained as
follows:
\begin{eqnarray*}
	\frac{\partial F}{\partial y} - \frac{d}{dx}\left(\frac{\partial F}{\partial \dot{y}}\right)
	&=& 0 - \frac{d}{dx}\left(\frac{1}{2}\left(1+\dot{y}^2\right)^{-\frac{1}{2}}2\dot{y}\right)\\
	&=& - \frac{d}{dx}\left(\dot{y}\left(1+\dot{y}^2\right)^{-\frac{1}{2}}\right)\\
	&=& 0
\end{eqnarray*}
Integrating with respect to $dx$
\[C = \dot{y}\left(1+\dot{y}^2\right)^{-\frac{1}{2}}\]
Squaring
\[C^2 = \dot{y}^2\left(1+\dot{y}^2\right)^{-1}\]
Collecting like terms and rearranging
\[\dot{y}^2 = \frac{C^2}{1-C^2}\]
\[\dot{y} = \pm\frac{C}{\sqrt{1-C^2}}\]
Therefore
\[\int dy = \int \pm\frac{C}{\sqrt{1-C^2}} dx\]
Which results in the following expression for $y$ where
$A=\pm\frac{C}{\sqrt{1-C^2}}$
\begin{equation}
	y = Ax + B
\end{equation}
Which is the equation for a straight line as expected.

\section{Deriving the Equations of Motion}
\label{sec:DerivingTheEquationsOfMotion}
\subsection{Newtonian Technique}
\label{sec:NewtonianTechnique}
Equations of motion for a single particle mass connected by two springs. The
springs connect the mass to two fixed points. We know the following.
\begin{itemize}
\item $k_{0}$ = the spring constant of spring 0.
\item $k_{1}$ = the spring constant of spring 1.
\item $m$ = the mass of the particle.
\item $x(0) = x_0$ = the position of the particle at the starting time.
\item $\dot{x}(0) = 0$ = the velocity of the particle at the starting time.
\end{itemize}
The last two items are known as the initial conditions of the system since they
describe the state of the system at a starting time. With the information above
a solution to the following
\begin{itemize}
\item $x(t)$ = the displacement of the particle from its resting position.
\end{itemize}
can be derived using Newtonian methods.
Let $F_{0}$ be the force on the particle due to spring 0.
\begin{equation}
	F_{0} = -k_{0}x
\end{equation}
Let $F_{1}$ be the force on the particle due to spring 1.
\begin{equation}
	F_{1} = -k_{1}x
\end{equation}
Then the total force on the particle is
\begin{equation}
	F = F_{0} + F_{1}
\end{equation}
and using Newton's law $F = ma$
\[F = (-k_{0} - k_{1})x = m\ddot{x}	\]
\[\ddot{x} + \frac{(k_{0} + k_{1})}{m}x = 0\]
Letting $K = \frac{(k_{0} + k_{1})}{m}$
\begin{equation}
	\label{eqn_motion}
	\ddot{x} + Kx = 0
\end{equation}
This is a standard 2nd order ODE that we can solve by letting $x = e^{rt}$.
Then $\ddot{x} = r^{2}e^{rt} = r^{2}x$ Therefore:
\[
	r^{2}x + Kx = 0
\]
and solving for r gives:
\[r = \pm\sqrt{-K} = \pm i \sqrt{K}\]
Therefore a general solution is:
\[
	x(t) = Ae^{i\sqrt{K}t} + Be^{-i\sqrt{K}t}	
\]
and by using $e^{i\theta} = cos(\theta) + isin(\theta)$ can also be written as
\[
	x(t) = Ccos(\sqrt{K}t) + Dsin(\sqrt{K}t)\\	
\]
and the derivative is
\[
	\dot{x}(t) = -\sqrt{K}Csin(\sqrt{K}t) + \sqrt{K}Dcos(\sqrt{K}t)
\]
Then using the initial conditions $x(0)=x_0$ and $\dot{x}(0) = 0$ implies that
$D=0$ and $C=x_0$ hence
\begin{equation}
	x(t) = x_0cos(\sqrt{K}t)
\end{equation}

\subsection{Lagrangian Technique}
Consider the same system described above. This time the equation of motion is
derived using Lagrangian methods. In order to do this using the Lagrangian we
need to know the kinetic and potential energy of the system in terms of $x(t)$.
\begin{equation}
	KE = \frac{1}{2}m\dot{x}^2
\end{equation}
\[
	PE = \frac{1}{2}k_0x^2 + \frac{1}{2}k_1x^2
\]
then 
\begin{eqnarray*}
L&=&KE - PE\\
	&=&\frac{1}{2}m\dot{x}^2 - \frac{1}{2}k_0x^2 - \frac{1}{2}k_1x^2
\end{eqnarray*}
by applying \ref{euler_lagrange}
\begin{equation}
	\frac{\partial L}{\partial x} = -k_0x - k_1x
\end{equation}
\begin{equation}
	\frac{d}{dt}\left(\frac{\partial L}{\partial \dot{x}}\right) = \frac{d}{dt}\left(m\dot{x}\right) = m\ddot{x}
\end{equation}
therefore
\begin{equation}
	\frac{\partial L}{\partial x} - \frac{d}{dt}\left(\frac{\partial L}{\partial \dot{x}}\right)\\
	= -k_0x - k_1x - m\ddot{x}\\
	= \ddot{x} + \frac{(k_{0} + k_{1})}{m}x 
	= 0
\end{equation}
Letting $K = \frac{(k_{0} + k_{1})}{m}$ we get
\begin{equation}
	\ddot{x} + Kx = 0
\end{equation}
This is the same as \ref{eqn_motion} and can be solved similarly. We have
derived the equation of motion for the spring mass system using Lagrangian
techniques.

