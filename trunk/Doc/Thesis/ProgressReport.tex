\documentclass[12pt]{report}
\usepackage{epsfig,amssymb,amsmath,amsfonts,url,pslatex,hyperref}
\newcommand{\urlBiBTeX}[1]{\href{#1}{}}
\begin{document}
\title{MSc Thesis Progress Report}
\date{\today}
\author{Sashan Kesniel Govender\\
The University of Auckland, \\
New Zealand.\\
\\
Master of Science} 
\maketitle
\newpage
The first half of the course was spent catching up with the literature related
to physically based modelling and animation \cite{Bourg, BarzelBarr}.
Specifically the SIGGRAPH course notes \cite{PBMNotes} by Witkin and Baraff on
physically based modelling were studied as well as Robin Otte's thesis
\cite{Otte}. A brief study of the basics of Lagrangian Dynamics was done
\cite{GoldStein, Wells}. Lagrangian methods allow the equations of motion for a
system to be derived using the concept of energy, as opposed to Newtonian
methods that derive the equations of motion using the concept of force. Some
revision of differential equations and ways to numerically solve them took place
\cite{NagleSaff, Eberly, Lee}.   

Some insight into constrained dynamics from the perspective of a mechanical
engineer is being gained via \cite{ComputationalDynamics}. Another interesting
paper is \cite{PracticalDynamics} and it discusses dynamical systems from a high
level perspective. Although it does not deal with geometric constraints
directly, it places a lot of emphasis on the practical benefits of a using mass
spring systems. Mass spring systems can be made to approximate a number of
physical entities, including a constrained particle system if the springs are
very stiff and highly damped. An implementation of a constrained particle system
was made according to \cite{PBMNotes, Otte} in which mass spring systems are not
considered viable. Implementing the system involved researching different matrix
libraries namely, \href{http://www.osl.iu.edu/research/mtl/}{MTL},
\href{http://www.oonumerics.org/blitz/}{Blitz++} and
\href{http://www.boost.org/libs/numeric/ublas/doc/index.htm}{uBLAS}. uBLAS was
choosen along with other auxiliary packages from the
\href{http://www.boost.org}{Boost} website. It was found that uBLAS, while it
provides support at a data structure level for a large number of varying matrix
types does not support the algorithms to solve matrix equations. For this reason
a biconjugate gradient solver was implemented \cite{NumRecipes}.

The implementation of the constraint based particle system using Newtonian
mechanics was validated by solving the equations of motion \cite{MyPhysicsLab}
of a pendulum and a double pendulum using Mathematica, plotting graphs of
position of the particles with respect to time and comparing the results with
the graphs generated by implementing a pendulum and double pendulum using
constraint based dynamics. Source code and thesis documentation (including the
latex file for this document) are hosted \href{http://edge.bkbits.net/}{here}.

Interesting methods for correcting the problem of numerical drift were
discussed, some more in depth than others. Numerical drift occurs because the
numerical integration techniques used won't give an exact solution to a
differential equation. Different ways to combat the problem were proposed:
verlet integrators, a scheme to minimize the error, and post correcting using a
mass spring system. The method using a mass spring system was implemented. 

Testing the code using unit tests where possible was enforced. The unit test
package used was the \href{http://www.boost.org/libs/test/doc/index.htm}{Boost
Test Library}. For example the RK4 algorithm implemented was tested using the
output given in \cite{NagleSaff}. The biconjugate gradient solver, implemented
from \cite{NumRecipes}, was tested by comparing its solution with that generated
by Mathematica. Unit tests provide some form of security when developing
applications.

\bibliographystyle{plain}
\bibliography{bibfile}
\end{document}

