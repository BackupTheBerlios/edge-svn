
\documentstyle{article}
\begin{document}
\title{\bfseries Thesis Proposal: Lagrangian Methods for Real-Time Simulation of Cars}
\date{\today}
\author{Sashan Kesniel Govender \\
The University of Auckland, Auckland, New Zealand \\
MSc Thesis} 
\maketitle

Physically based modeling for simulating the motion of cars in the field of computer graphics is predominantly based on Newtonian methods. The majority of the literature \cite{Otte, Bourg, Beckman, Genta} I have encountered related to physically modeling cars and rigid bodies deals with the problem using Newtonian mathematical methods. It is well known that these problems can also be solved using Lagrangian methods. This thesis aims to investigate the application of Lagrangian mathematical methods to derive a physically based model suitable for the simulation of a car on a computer.

The main goal of this thesis is to show that Lagrangian methods can be used to develop a model suitable for the simulation of a car on a computer. This implies the assimilation of previous work on the dynamics of rigid body system, the dynamics of cars, Lagrangian methods and the synthesis thereof to generate a framework for the real-time simulation of a car on a computer. The car model will hopefully cater for the following physical effects: the effects of acceleration on a car, the effects of aerodynamic and rolling resistance, low and high speed cornering and suspension. Another goal is to implement a simulator that illustrates the concepts discussed. A possible side effect resulting from this thesis will be a comparison between the Newtonian and Lagrangian methods for physically based modeling. 

The simulator will be written in C++. Mathematica will be used for high level experimentation and testing. Any geometric data that needs to be generated can be done using 3D Studio Max or Wings3D.
\bibliographystyle{amsalpha}
\bibliography{bibfile}
\end{document}
